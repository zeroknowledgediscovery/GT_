\documentclass[onecolumn, compsoc,10pt]{IEEEtran}
\let\labelindent\relax
\usepackage{enumitem}
\usepackage{etex}
\usepackage{amssymb,amsfonts,amsmath,amsthm}
\usepackage{graphicx}
 \usepackage[usenames,x11names, dvipsnames, svgnames]{xcolor}
\usepackage{amsmath,amssymb}
\usepackage{dsfont}
\usepackage{mathrsfs}
\usepackage{hyperref}
\hypersetup{
    colorlinks=true,
    linkcolor=black,
    citecolor=MediumBlue,
    filecolor=black,
    urlcolor=DodgerBlue4,
    breaklinks=false,
%linkbordercolor=red,% hyperlink borders will be red
  %pdfborderstyle={/S/U/W 1}% border style will be underline of width 1pt
}
\usepackage{array}
%\usepackage{multirow}    
%\usepackage[T1,euler-digits]{eulervm}
%\usepackage{times}
%\usepackage{pxfonts}
\usepackage{tikz}
\usepackage{pgfplots}
\usetikzlibrary{shapes,calc,shadows,fadings,arrows,decorations.pathreplacing,automata,positioning}
\usetikzlibrary{external}
\usetikzlibrary{decorations.text}
\tikzexternalize[prefix=./Figures/External/]% activate externalization!
\tikzexternaldisable
%\addtolength{\voffset}{.1in}  
\usepackage{geometry}
\geometry{a4paper, left=.65in,right=.65in,top=.8in,bottom=0.8in}

\addtolength{\textwidth}{-.1in}    
\addtolength{\hoffset}{.05in}    
\addtolength{\textheight}{.1in}    
\addtolength{\footskip}{0in}    
\usepackage{rotating}
 \definecolor{nodecol}{RGB}{240,240,220}
 \definecolor{nodeedge}{RGB}{240,240,225}
  \definecolor{edgecol}{RGB}{130,130,130}
    \tikzset{%
fshadow/.style={      preaction={
         fill=black,opacity=.3,
         path fading=circle with fuzzy edge 20 percent,
         transform canvas={xshift=1mm,yshift=-1mm}
       }} 
}
\usetikzlibrary{pgfplots.dateplot}
 \usetikzlibrary{patterns}
\usetikzlibrary{decorations.markings}
\usepackage{fancyhdr}
\usepackage{mathtools}
\usepackage{datetime}
\usepackage{comment}
%% ## Equation Space Control---------------------------
\def\EQSP{2pt}
\newcommand{\mltlne}[2][\EQSP]{\begingroup\setlength\abovedisplayskip{#1}\setlength\belowdisplayskip{#1}\begin{equation}\begin{multlined} #2 \end{multlined}\end{equation}\endgroup}
\newcommand{\cgather}[2][\EQSP]{\begingroup\setlength\abovedisplayskip{#1}\setlength\belowdisplayskip{#1}\begin{gather} #2 \end{gather}\endgroup}
\newcommand{\cgathers}[2][\EQSP]{\begingroup\setlength\abovedisplayskip{#1}\setlength\belowdisplayskip{#1}\begin{gather*} #2 \end{gather*}\endgroup}
\newcommand{\calign}[2][\EQSP]{\begingroup\setlength\abovedisplayskip{#1}\setlength\belowdisplayskip{#1}\begin{align} #2 \end{align}\endgroup}
\newcommand{\caligns}[2][\EQSP]{\begingroup\setlength\abovedisplayskip{#1}\setlength\belowdisplayskip{#1}\begin{align*} #2 \end{align*}\endgroup}
\newcommand{\mnp}[2]{\begin{minipage}{#1}#2\end{minipage}} 
%% COLOR DEFS------------------------------------------
\newtheorem{thm}{Theorem}
\newtheorem{cor}{Corollary}
\newtheorem{lem}{Lemma}
\newtheorem{prop}{Proposition}
\newtheorem{defn}{Definition}
\newtheorem{example}{Example}
\newtheorem{rem}{Remark}
\newtheorem{notn}{Notation}
%%------------PROOF INCLUSION -----------------
\def\NOPROOF{Proof omitted.}
\newif\ifproof
\prooffalse % or \draftfalse
\newcommand{\Proof}[1]{
\ifproof
\begin{IEEEproof}
#1\end{IEEEproof}
\else
\NOPROOF
\fi
 }
%%------------ -----------------
\newcommand{\DETAILS}[1]{#1}
%%------------ -----------------
% color commands------------------------
\newcommand{\etal}{\textit{et} \mspace{3mu} \textit{al.}}
% \renewcommand{\algorithmiccomment}[1]{$/** $ #1 $ **/$}
\newcommand{\vect}[1]{\textbf{\textit{#1}}}
\newcommand{\figfont}{\fontsize{8}{8}\selectfont\strut}
\newcommand{\hlt}{ \bf \sffamily \itshape\color[rgb]{.1,.2,.45}}
\newcommand{\pitilde}{\widetilde{\pi}}
\newcommand{\Pitilde}{\widetilde{\Pi}}
\newcommand{\bvec}{\vartheta}
\newcommand{\algo}{\textrm{\bf\texttt{GenESeSS}}\xspace}
\newcommand{\xalgo}{\textrm{\bf\texttt{xGenESeSS}}\xspace}
\newcommand{\FNTST}{\bf }
\newcommand{\FNTED}{\color{darkgray} \scriptsize $\phantom{.}$}
\renewcommand{\baselinestretch}{.95}
\newcommand{\sync}{\otimes}
\newcommand{\psync}{\hspace{3pt}\overrightarrow{\hspace{-3pt}\sync}}
%\newcommand{\psync}{\raisebox{-4pt}{\begin{tikzpicture}\node[anchor=south] (A) {$\sync$};
%\draw [->,>=stealth] ([yshift=-2pt, xshift=2pt]A.north west) -- ([yshift=-2pt]A.north east); %\end{tikzpicture}}}
\newcommand{\base}[1]{\llbracket #1 \rrbracket}
\newcommand{\nst}{\textrm{\sffamily\textsc{Numstates}}}
\newcommand{\HA}{\boldsymbol{\mathds{H}}}
\newcommand{\eqp}{ \vartheta }
\newcommand{\entropy}[1]{\boldsymbol{h}\left ( #1 \right )}
\newcommand{\norm}[1]{\left\lVert #1 \right\rVert}%
\newcommand{\abs}[1]{\left\lvert #1 \right\rvert}%
\newcommand{\absB}[1]{\big\lvert #1 \big\rvert}%
% #############################################################
% #############################################################
% PREAMBLE ####################################################
% #############################################################
% #############################################################
% \usepackage{pnastwoF}
\DeclareMathOperator*{\argmax}{argmax}
\newcommand{\ND}{ \mathcal{N}  }
\usepackage[linesnumbered,ruled,vlined,noend]{algorithm2e}
\newcommand{\captionN}[1]{\caption{\color{darkgray} \sffamily \fontsize{8}{10}\selectfont #1  }}
\newcommand{\btl}{\ \textbf{\small\sffamily bits/letter}}
\usepackage{txfonts}
%\usepackage{ccfonts}
%%% save defaults
\renewcommand{\rmdefault}{phv} % Arial
\renewcommand{\sfdefault}{phv} % Arial
\edef\keptrmdefault{\rmdefault}
\edef\keptsfdefault{\sfdefault}
\edef\keptttdefault{\ttdefault}

%\usepackage{kerkis}
\usepackage[OT1]{fontenc}
\usepackage{concmath}
%\usepackage[T1]{eulervm}
%\usepackage[OT1]{fontenc}
%%% restore defaults
\edef\rmdefault{\keptrmdefault}
\edef\sfdefault{\keptsfdefault}
\edef\ttdefault{\keptttdefault}
\tikzexternalenable
% ##########################################################
\tikzfading[name=fade out,
            inner color=transparent!0,
            outer color=transparent!100]
%###################################
\newcommand{\xtitaut}[2]{
\noindent\mnp{\textwidth}{
\mnp{\textwidth}{\raggedright\Huge \bf \sffamily #1}

\vskip 1em

{\bf \sffamily #2}
}
\vskip 2em
}
%###################################
%###################################
\tikzset{wiggle/.style={decorate, decoration={random steps, amplitude=10pt}}}
\usetikzlibrary{decorations.pathmorphing}
\pgfdeclaredecoration{Snake}{initial}
{
  \state{initial}[switch if less than=+.625\pgfdecorationsegmentlength to final,
                  width=+.3125\pgfdecorationsegmentlength,
                  next state=down]{
    \pgfpathmoveto{\pgfqpoint{0pt}{\pgfdecorationsegmentamplitude}}
  }
  \state{down}[switch if less than=+.8125\pgfdecorationsegmentlength to end down,
               width=+.5\pgfdecorationsegmentlength,
               next state=up]{
    \pgfpathcosine{\pgfqpoint{.25\pgfdecorationsegmentlength}{-1\pgfdecorationsegmentamplitude}}
    \pgfpathsine{\pgfqpoint{.25\pgfdecorationsegmentlength}{-1\pgfdecorationsegmentamplitude}}
  }
  \state{up}[switch if less than=+.8125\pgfdecorationsegmentlength to end up,
             width=+.5\pgfdecorationsegmentlength,
             next state=down]{
    \pgfpathcosine{\pgfqpoint{.25\pgfdecorationsegmentlength}{\pgfdecorationsegmentamplitude}}
    \pgfpathsine{\pgfqpoint{.25\pgfdecorationsegmentlength}{\pgfdecorationsegmentamplitude}}
  }
  \state{end down}[width=+.3125\pgfdecorationsegmentlength,
                   next state=final]{
     \pgfpathcosine{\pgfqpoint{.15625\pgfdecorationsegmentlength}{-.5\pgfdecorationsegmentamplitude}}
     \pgfpathsine{\pgfqpoint{.15625\pgfdecorationsegmentlength}{-.5\pgfdecorationsegmentamplitude}}
  }
  \state{end up}[width=+.3125\pgfdecorationsegmentlength,
                 next state=final]{
     \pgfpathcosine{\pgfqpoint{.15625\pgfdecorationsegmentlength}{.5\pgfdecorationsegmentamplitude}}
     \pgfpathsine{\pgfqpoint{.15625\pgfdecorationsegmentlength}{.5\pgfdecorationsegmentamplitude}}
  }
  \state{final}{\pgfpathlineto{\pgfpointdecoratedpathlast}}
}
%###################################
%###################################
\newcolumntype{L}[1]{>{\rule{0pt}{2ex}\raggedright\let\newline\\\arraybackslash\hspace{0pt}}m{#1}}
\newcolumntype{C}[1]{>{\rule{0pt}{2ex}\centering\let\newline\\\arraybackslash\hspace{0pt}}m{#1}}
\newcolumntype{R}[1]{>{\rule{0pt}{2ex}\raggedleft\let\newline\\\arraybackslash\hspace{0pt}}m{#1}}




\newcommand{\drhh}[8]{
\begin{axis}[semithick,
font=\bf \sffamily \fontsize{7}{7}\selectfont,
name=H2,
at=(#4), anchor=#5,
xshift=.3in,
yshift=-.3in,
width=\WDT, 
height=\HGT, 
title={{\LARGE G } ROC area distribution (Out-of-sample)}, 
title style={align=right, },legend cell align=left,
legend style={ xshift=3.5in, yshift=-.6in, draw=white, fill= gray, fill opacity=0.2, 
text opacity=1,},
axis line style={black!80, opacity=0,   thick,,ultra thin, rounded corners=0pt},
axis on top=false, 
xlabel={ROC area},
ylabel={probability},
ylabel style={yshift=-.25in},
xlabel style={yshift=.1in},
grid style={dashed, gray!50},
%grid,
axis background/.style={top color=gray!1,bottom color=gray!2},
enlargelimits=false, 
scale only axis=true,
ymin=0,
%xmin=.7,xmax=1.0,
ylabel style={yshift=.05in},
major tick length=0pt,yticklabel style={/pgf/number format/fixed,/pgf/number format/precision=2},xticklabel style={/pgf/number format/fixed,/pgf/number format/precision=2},
#7,
 ]
\addplot [
    fill=#8,
    thick,
    draw=white,
    opacity=1,
    hist={density,bins=10},
] table [y index=#3] {#1};
% \addlegendentry{$\Delta$ ROC};
\addplot [very thick, Red2,, opacity=.95] gnuplot [raw gnuplot] {plot '#1' u #2:(1./#6.) smooth kdensity};
%
%\draw[thin,black ] (axis cs:.89291,\pgfkeysvalueof{/pgfplots/ymin}) -- (axis cs:.89291,\pgfkeysvalueof{/pgfplots/ymax}) node [midway,right, pos=0.2] {89.3\%};
% \addlegendentry{kde};
\end{axis}
}


\newcommand{\erhh}[6]{
  \begin{axis}[semithick,
font=\bf \sffamily \fontsize{7}{7}\selectfont,
name=H2,
at=(#3), anchor=#4,
xshift=.3in,
yshift=-.3in,
width=\WDT, 
height=\HGT, 
title style={align=center, },legend cell align=left,
legend style={ xshift=3.5in, yshift=-.6in, draw=white, fill= gray, fill opacity=0.2, 
text opacity=1,},
axis line style={black!80, opacity=0,   thick,,ultra thin, rounded corners=0pt},
axis on top=false, 
xlabel={ROC area},
ylabel={probability},
ylabel style={yshift=-.25in},
xlabel style={yshift=.1in},
grid style={dashed, gray!50},
%grid,
axis background/.style={top color=gray!1,bottom color=gray!2},
enlargelimits=false, 
scale only axis=true,
%ymin=0, 
%xmin=.7,xmax=1.0,
ylabel style={yshift=.05in},
major tick length=0pt,yticklabel style={/pgf/number format/fixed,/pgf/number format/precision=2},xticklabel style={/pgf/number format/fixed,/pgf/number format/precision=2},
#5,
 ]
    \addplot[semithick, #6]
    table[x expr=(\coordindex+1),y expr=(\thisrowno{#2})] {#1};
    % \addlegendentry{Cullman, Alabama};
  \end{axis}
}
%################################################
%################################################
%################################################
%################################################
\def\DISCLOSURE#1{\def\disclosure{#1}}
\DISCLOSURE{\raisebox{15pt}{$\phantom{XxxX}$This sheet contains proprietary information 
 not to be released to third parties except for the explicit purpose of evaluation.}
}
 
\usepackage{geometry}
\geometry{a4paper, left=.7in,right=.75in,top=.8in,bottom=0.65in}
\usepackage{textcomp}
\usepackage{colortbl}
% \usepackage{subfigure}
\usepackage{array}
\usepackage{courier}
\usepackage{wrapfig}
\usepackage{pifont} 
\usepackage{subfig} 
\usetikzlibrary{chains,backgrounds}
\usetikzlibrary{intersections}
\usepackage[super]{cite}  
\usepackage{setspace} 
\makeatletter \renewcommand{\@citess}[1]{\raisebox{1pt}{\textsuperscript{[#1]}}} \makeatother
\usepackage{xstring}
\usepackage{xspace}
\captionsetup[subfigure]{labelformat=empty}
\makeatletter
\renewcommand\section{\@startsection {section}{1}{\z@}%
  {-2ex \@plus -1ex \@minus -.2ex}%
  {1ex \@plus.1ex}%
  {\LARGE\bfseries\scshape}}

\def\thesectiondis{\thesection} \def\thesubsectiondis{\thesectiondis\alph{subsection}.} \def\thesubsubsectiondis{\thesubsectiondis\roman{subsubsection}.} %\def\theparagraphdis{\thesubsubsectiondis\arabic{paragraph}.}

\makeatother
\makeatletter
\pgfdeclareradialshading[tikz@ball]{ball}{\pgfqpoint{-10bp}{10bp}}{%
  color(0bp)=(tikz@ball!30!white);
  color(9bp)=(tikz@ball!75!white);
  color(18bp)=(tikz@ball!90!black);
  color(25bp)=(tikz@ball!70!black);
  color(50bp)=(black)}
\makeatother
\newcommand{\tball}[1][CadetBlue4]{${\color{#1}\Large\boldsymbol{\blacksquare}}$}
\renewcommand{\baselinestretch}{1.1075}
\renewcommand{\captionN}[1]{\caption{\color{CadetBlue4!80!black} \sffamily \fontsize{8}{9}\selectfont #1  }}
\tikzexternaldisable 
\parskip=7pt
\parindent=0pt
\newcommand{\Mark}[1]{\textsuperscript{#1}}
% \lhead{\sf\footnotesize \color{DodgerBlue4!70!black}\today}
\pagestyle{fancy} 
\def\COLA{black}
% ###################################
\cfoot{}
% \cfoot{}
\rhead{\scriptsize\bf\sffamily\thepage}
\lhead{\scriptsize\bf\sffamily DARPA GT \today}
\newcommand{\partxt}{\bf\sffamily\itshape}
% ############################################################
\newif\iftikzX
\tikzXtrue 
\tikzXfalse
\tikzexternalenable
\def\jobnameX{atd}
\newcommand{\SPX}[1][50pt]{\vspace{#1}}
% ############################################################
\newcommand{\incomplete}{\colorbox{Red1!80}{\textbf{\footnotesize\color{white}(incomplete section)}}}
\def\FWN{\textbf{\small FWN}\xspace}
% ############################################################
% ##################################
% ##################################

\def\TITLE{Post Process Cynet Runs }
\author{}
\date{June 2018}

\begin{document}
% 
% 
\xtitaut{\bf\sffamily \color{black!90!DodgerBlue1}  \fontsize{15}{16}\selectfont 
\vskip 1em
  {\TITLE \\
    % \hrule 
\small ishanu chattopadhyay
  }
}{}
\vspace{-15pt}   


\section{Q1 \& Q2: Pred. Visitors \& Meetings}


\begin{itemize}
\item 2 scenarios (A. No change, B. Closed sites)
\item Need confidence bounds for Q1 (prediction of visits to rec sites)
\end{itemize}

\def\N{\mathcal{N}}
\def\ss{s^\star}
\def\S{\mathscr{S}}

\subsection{Scenario A: No Change}

Let $\{s^0\}$ be the set of cynet-run sites, and let $\ss$ be a target site.

Let $\mathcal{N}_k(\ss) \subset \{s^0\}$ be the set of $k$ closest neighboring sites of $\ss$ in $\{s^0\}$. Note, ``closest'' implies we are using a specific distance metric, the choice of which is described later.

Step 1. Compute predictions for all sites $\{s^0\}$, by computing the regressors on FTX files as done before. The predicted time series for site $s$ is denoted as $s_t$.

Step 2. We then use: (Denoting a neighboring site as $s^i$)
\cgather{
\ss_t = \sum_{\mathclap{s^i \in\mathcal{N}_k(\ss) }} \alpha_i s_t^i
}
where
\cgather{\label{updateq}
\alpha_i = \frac{\displaystyle 1/\theta(s^i,\ss)}{\displaystyle \sum_{\mathclap{s^i \in\mathcal{N}_k(\ss) }}1/\theta(s^i,\ss)}
}

Note: we still need to choose $k$. We set it to some small value in the range $[10,35]$.

\subsection{Distance Metric}

Three different distance metrics may be used. We will try all of them:
\begin{enumerate}
\item Euclidean
\item Travel distance
\item Learned from prediction behavior (discussed later)
\end{enumerate}

\subsection{Scenario B: Site Closing}

Let $\S$ be the set of all rec sites.

Let $s^c$ be a closed site, and let $\N_k^\S(s^c) \subset \S$ be the set of $k$ neighbors of $s^c$. Note this is the set of neighbors among \textbf{all} rec sites, which is different from $\N_k(\ss)$.


Step 1. Predict $s_t$ for all $s \in \N_k^\S(s^c)$ using the approach in the previous section.

Step 2. Compute:
\cgather{
\forall s^i \in \N_k^\S(s^c), \ \beta_i =  \frac{\displaystyle 1/\theta(s^i,s^c)}{\displaystyle \sum_{\mathclap{s^i \in\mathcal{N}_k^\S(s^c) }}1/\theta(s^i,s^c)}
}

Step 3. Predict $s^c_t$ using the approach in the previous section

Step 4. Change for  sites  $s^i \in \N_k(s^c)$ as:
\cgather{
\delta s^i_t  = \beta_i s^c_t
}

Step 5. Do this for all closed sites (which updates a bunch of neighboring rec sites). Let the set of sites updated be denoted as $U \subset \S$. We know, the total change for each site in $U$, which is denoted as $\delta s^u_t$ for $s^u \in U$.

Step 6. Update sites in $\{s^0\}$ as follows:
For each $s^u \in U$, compute the $\alpha$-coefficients as in previous section. In other words, 
\cgather{\label{updateq2}
\alpha_i^u = \frac{\displaystyle 1/\theta(s^i,s^u)}{\displaystyle \sum_{\mathclap{s^i \in\mathcal{N}_k(s^u) }}1/\theta(s^i,s^u)}}
Then, for each $s^u \in U$:
\cgather{
s_t^i \leftarrow s_t^i + \sum_{s^u \in U}\alpha_i^u \delta s^u_t
} 

Step 6.  Update target sites using approach in previous section.


The last two sections are sufficient to answer both Q1 and Q2 for both scenarios. %Now, we define the computation of confidence bounds.
\textbf{Confidence Bounds Needs for Q1}

\section{Q3 : Individual List of REC Sites Visited}

\subsection{Scenario A}

\subsubsection{Individuals Already in Cynet Run}

Step 1. Compute Regressor to predict $3$ variables for all uid that were already in cynet run1:
\begin{itemize}
\item distHome
\item distWork
\item RecSiteVisit
\end{itemize}

Step 2. Identify which recsites were actually visited (when RecSiteVisit predicts a visit), by looking up the nearest RecSite that satisfies the distHome and distWork distances at that time.

Step 3. Compute list of RecSites visited (in prediction period) by each individual

\subsubsection{Individuals Not  in Cynet Run}

\textit{Please message me if there are indeed any individuals who are not included in the cynet run? }


\def\H{H^0}
\def\hh{h^\star}

Let $H^0$ be the set of individuals in the cynet run.

Let  $h^\star$ be an individual not in the run for whom we have to make predictions
Let $\omega(h)$ be the coordinate of the workplace for individual $h$, and $\sigma(h)$ be the coordiante of the home.

Additionaly, let $h_t^\omega, h_t^\sigma$ be the predictions of distWork and distHome  for indivdual $h$.

Assuming Step 1. above is done.

Step 2. For each day $t$ in prediction period, set up and solve the following 2  
regression problems:
\cgather{
\textrm{Given } h_t^\omega, h_t^\sigma \xrightarrow{predict} distHome \textrm{ (when at Rec site on day $t$ )}\\
\textrm{Given }h_t^\omega, h_t^\sigma \xrightarrow{predict} distWork \textrm{ (when at Rec site on day $t$)}
}

Step 3. Use regressors computed in Step 2. to estimate distHome, distWork for $\hh$.

Step 4. Compute list of sites 



\subsection{Scenario B: Sites Close}

We simply select the next closest site when computing the REC Site Ids


\section{Q4: Average Number of Rec Site Visits Per Day}

Step 1. Compute from RecSiteVisit Predictions directly on individuals in cynet run.


\textbf{Confidence Bounds Needs for Q4}


\section{Confidence Bounds Calculation}

Step 1. Vary the model slection at Cynet prediction stage for all calculations, to generate a set of different predictions (time series levels)

Step 2. Use this bank of time series to compute confidence bounds.


\clearpage



\end{document}








\section{Preliminary Notions \& Notations}
\begin{notn}[A Priori Sites]
Let 
\cgather{
Q=\{ q_1 , \cdots , q_i, \cdots , q_m\}
}
 denote the set of nodes or sites existing a priori. Note that the location of  each $q_i$ is specified by its X and Y coordinates. 
\end{notn}
\begin{notn}[Euclidean Distance]
The Euclidean distance between sites $q_i,q_j$ is denoted by $d(q_i,q_j)$. This is simply:
\cgather{
d(q_i,q_j) = \sqrt{(x_i-x_j)^2 + (y_i -y_j)^2 }
}
where $x_i,x_j$ and $y_i,y_j$ are the X and Y coordinates of $q_i,q_j$ repectively. 
\end{notn}
\begin{defn}[Scaling Of Probability distribution]
For a given probability distribution $\wp$, and a scalar $c \in \mathbb{R}$, we define the scaled distribution $\wp'$ as:
\cgather{
\wp'_i = c \odot \wp = \frac{\wp_i^c}{\sum_i \wp_i^c}
}
\end{defn} 
\begin{example}
Let $\wp=[0.1 \  0.2 \  0.7 ]$ and $c=0.2$. Then 
\cgather{
0.2 \star \wp = [ 0.1^{0.2} \ 0.2^{0.2} \ 0.7^{0.2} ] \times \frac{1}{ 0.1^{0.2} + 0.2^{0.2} + 0.7^{0.2}}
}
\end{example}
\begin{defn}[Addition Of Probability distribution]
For a given probability distributions $\wp,\wp'$, the sum is defined as:
\cgather{
(\wp \oplus \wp')_i =  \frac{\wp_i \wp_i'}{\sum_i \wp_i \wp_i'}
}
\end{defn} 
\begin{example}
Let $\wp=[0.1 \  0.2 \  0.7 ]$ and $\wp'=[0.3 \  0.3 \  0.4]$. Then, 
\cgather{
(\wp \oplus \wp')_i = [ 0.1\times 0.3 \mspace{15mu} 0.2\times0.3 \mspace{15mu}  0.7 \times 0.4 ] \times \frac{1}{ (0.1\times 0.3) + (0.2\times0.3) + (0.7 \times 0.4)}
}
\end{example}

It follows from the above definitions that for any distribution $\wp$:
\cgather{
\wp \oplus (-1)\wp = u
}
where u is the uniform distribution on the same alphabet.


\begin{defn}[Hybrid Scaling ]
Given a tuple $(c,\wp)$ where $c \in \mathbb{R}$ and $\wp$ is a probability distribution, we define hybrid scaling by $\theta$ as follows:
\cgather{
\theta \odot (c, \wp) = (\theta c , \theta \odot \wp)
}
\end{defn}

\begin{defn}[Normalized Coordinates]
For a  site $P$ (new or existing a priori), its normlized coordinates $\Theta(P)$ is defined as:
\cgather{
\forall i \in \{1,\cdots, m\}, \Theta(P)_i = \frac{d(P,q_i)}{\sum_{i=1}^m d(P,q_i)  }
}
where $q_i$ are the a priori sites.
\end{defn}

\begin{defn}[Normalized Inverse Coordinates ]
We define the normalized inverse coordinates as:
\cgather{
\forall i \in \{1,\cdots, m\}, \Theta^\star(P)_i = \frac{1}{\Theta(p)_i}
}
\end{defn}

Recall the FTX files generated in the course of the \textit{cynet} simulation.
Each line in each FTX file is actually a pair $(\gamma,\wp)$, where $\gamma$ is a scalar, and $\wp $ is a probability distribution.

\begin{notn}
Let the FTX file for site $q_i$ at time $t$ be denoted as $F_i^t$.
\end{notn}

\begin{defn}[FTX Scaling]
Given the FTX file $F_i^t$, and a scalar $c$, the scaled FTX file $c \odot F_i^t$ is defined as:
\cgather{
(\gamma, \wp) \in F_i^t \Rightarrow (c\gamma, c \odot \wp) \in c \odot F_i^t
}
\end{defn}
\begin{notn}
Given the FTX file $F_i^t$, we denote the regressor output as:
\cgather{
\mathcal{R}_i(F_i^t)
}
This is the predicted number of visits at time $t$ at site $q_i$.
\end{notn}

Now we are ready to specify the addition calculation. 

\section{Addition of New Nodes}
We calculate the visitors as a function of time to newly added nodes. And also the changes to the priorly eisting nodes due to the addition of one or more new nodes.

Let $s$ be a new site added. Then we calcualte the number of \textit{pre-visits} as:
\cgather{
\hat{u}_i(s) = \mathcal{R}_i \left (\frac{1}{\Theta(s)_i} \odot F_i^t \right )
}

Allowing for the possibility of adding mutiple new sites $s \in S$, we compute:

\cgather{
u_i(s) = \frac{\hat{u}_i(s) }{ \mathcal{R}_i ( F_i^t)+   \hat{u}_i(s) }
}
And finally, we specify that  the visit to new site $s$ at time $t$ is given by:
\cgather{
  u(s) = \sum_{q_i \in Q} u_i(s) \mathcal{R}_i ( F_i^t)
}
And the updated  visits to site $q_i$ is given by:
\cgather{
\forall q_i \in Q, u(q_i) =  \mathcal{R}_i ( F_i^t) - \sum_{s \in S} u_i(s) \mathcal{R}_i ( F_i^t)
}

Next the subtraction algorithm.
\section{Removal of Nodes}

Note that the normalized coordinates can be calculated for any of the existing nodes as well. Assume that $q_j \in Q$ is to be removed. Then, the updated visits to site $q_i$ is given by: 
\cgather{
\forall q_i \neq q_j \in Q,  u(q_i) = \mathcal{R}_i ( F_i^t)  + \frac{\mathcal{R}_j \left (\frac{1}{\Theta(q_j)_i} \odot F_j^t \right )}{\sum_i \mathcal{R}_j \left (\frac{1}{\Theta(q_j)_i} \odot F_j^t \right )}  \mathcal{R}_j ( F_j^t)
}
where we assume that $\mathcal{R}_i(x) \rightarrow 0$ as $\vert x \vert  \rightarrow \infty$.
\end{document}
